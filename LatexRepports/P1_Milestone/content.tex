

















\subsection*{Causality or Correlation? Unlikely Trends in the New Yorker Caption Contest}
Explore strange or unexpected correlations between different trends in the Caption Contest that appear unrelated at first sight. For example, is there a connection between the seasons and jokes about cats? How do jokes about politicians affect the number of people depicted in the cartoon? Does the presence of certain objects (like cars, glasses, or telephones) correlate with particular humour strategies? We will systematically identify such patterns by cross-analysing captions, metadata, and funniness scores across distinct spheres (visual elements, themes, cultural references). The focus is not necessarily to establish causal relationships, but rather to highlight these surprising overlaps and ask whether they reflect deeper societal attitudes, hidden cognitive associations, or simply coincidences. The outcome may provide both serious insights into how people connect ideas in humour and amusing examples of “funny but meaningless” correlations.

\subsection*{Politics and Humour}
This study looks at whether the political views of the platform’s owners, editors, or sponsors influence the themes (and their popularity) of the Caption Contest. We will explore if captions tend to align more with one political side (for example Democratic vs. Republican) and whether opposing groups are more often shown with negative connotations. Using tags, description and rating we will check for patterns in how political figures, parties, or social issues appear, and ask whether these trends reflect editorial influence, platform politics, or just the general mood of the participants.

\subsection*{What do people like and how?}
Which objects, themes, and settings make captions more or less funny? We will list the most frequent objects in the cartoons and study how their presence (alone or in combination) relates to caption ratings. The meaning of an object often changes with context: a chair in an office suggests routine, while the same chair in an empty room with a rope above suggests something darker. By comparing visual elements with funniness scores, we aim to reveal which combinations audiences enjoy most and how context shapes the way humour is received.





